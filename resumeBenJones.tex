%%%%%%%%%%%%%%%%%%%%%%%%%%%%%%%%%%%%%%%%% 
% Medium Length Professional CV
% LaTeX Template
% Version 2.0 (8/5/13)
% 
% This template has been downloaded from:
% http://www.LaTeXTemplates.com
% 
% Original author:
% Trey Hunner (http://www.treyhunner.com/)
% Adapted by:
% Christian Moomaw (cdm@phosfor.us)
% Ben Jones (benkaijones@gmail.com)
% 
% Important note:
% This template requires the resume.cls file to be in the same directory as the
% .tex file. The resume.cls file provides the resume style used for structuring the
% document.
% 1
%%%%%%%%%%%%%%%%%%%%%%%%%%%%%%%%%%%%%%%%% 

\documentclass{resume} % Use the custom resume.cls style

\usepackage[left=0.75in,top=0.6in,right=0.75in,bottom=0.6in]{geometry} % Document margins

\name{Benjamin Kai Jones} % Your name
\address{676 Nissa Court, Auburn, AL 36830} % Your address
\address{bkj0005@auburn.edu -- 901-219-3263} % Your phone number and email

\begin{document}

\begin{rSection}{Education}

   \begin{rSubsection}{Auburn University}{May 2020 - December 2022}{Master of Science in Mechanical Engineering}{Auburn, AL}
  \item GPA: 4.0
  \item \textbf{Master's Thesis}: Collaborative Architectures for Relative Position Estimation of Ground Vehicles with UWB Ranging and Vehicle Dynamic Models
  \end{rSubsection}

 \begin{rSubsection}{Mississippi State University}{August 2016 - May 2020}{Bachelor of Science in Mechanical Engineering}{Starkville, MS}
  \item GPA: 3.97, Summa Cum Laude
  \end{rSubsection} 

\end{rSection}

\begin{rSection}{Relevant Coursework}
	Fundamentals of Navigation, Optimal Estimation and Control, Fundamentals of GPS, Nonlinear Systems and Control, Software for Sensors, Advanced Dynamics, Machine Learning

\end{rSection}

\begin{rSection}{Conference Presentations}
  \textbf{Jones, B.}, et al. ``Utilizing a Vehicle Dynamics Model for Ground-Vehicle Relative Position Estimation with UWB Ranges,'' Modeling Estimation and Controls Conference (MECC), Jersey City, New Jersey, 2022.
\end{rSection}

\begin{rSection}{Experience}

	\begin{rSubsection}{Radiance Technologies}{January 2023 - Present}{PNT Engineer}{Huntsville, AL}
	 \item Assisting with start up of Positioning, Navigation, and Timing laboratory
		\item Contributing to a small team exploring resiliency methods for PNT
	\end{rSubsection}

  \begin{rSubsection}{Auburn University GPS and Vehicle Dynamics Laboratory}{May 2020 - December 2022}{Graduate Research Assistant}{Auburn, AL}
  \item Research and develop navigation and estimation algorithms within MATLAB, C++, and ROS
  \item Implemented a real-time relative position estimation algorithm for autonomous vehicle convoys without a GPS reference
  \item Experienced in Kalman filtering with GPS, IMUs, Ultra-wideband radios, and vehicle models in both simulation and hardware
  \item \textbf{Projects}: Multi-agent collaborative aerial navigation, IMU on-line alignment and calibration, range-only relative position estimation
  \end{rSubsection}

  \begin{rSubsection}{CFD Research Corporation}{May - August 2019}{Engineering Intern}{Huntsville, AL}
  \item Developed and delivered a coherent testing environment for industry customer
  \item Solved trajectory and radar integration within MATLAB and Simulink
  \item Assisted ongoing development of hypersonic analysis tool within Modern Fortran and Python
  \end{rSubsection}
  
  
 %% \begin{rSubsection}{MSU Institute for Clean Energy Technology}{2018 - 2020 Academic Years}{Undergraduate Researcher}{Starkville, MS}
  %%\item Operated and Analyzed auto-ignition phenomena of oxygenated biofuels within rapid compression cycles
  %%\item Research presented at MSU undergraduate research symposium
  %%\end{rSubsection}
  
  
\end{rSection}

\begin{rSection}{Technical Strengths}

  \begin{tabular}{ @{} >{\bfseries}l @{\hspace{6ex}} l }
  	Active Security Clearance (Secret) \\
    Programming Languages & MATLAB, C++, Python \\
    Software & GIT, Robot Operating System (ROS), SOLIDWORKS, Abaqus \\
  \end{tabular}

\end{rSection}


\end{document}
